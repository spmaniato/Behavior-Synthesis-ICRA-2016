%%%%%%%%%%%%%%%%%%%%%%%%%%%%%%%%%%%%%%%%%%%%%%%%%%%%%

\documentclass[letterpaper, 10 pt, conference]{ieeeconf}	% Comment this line out if you need a4paper

\IEEEoverridecommandlockouts          	% This command is only needed if you want to use the \thanks command
\overrideIEEEmargins                  	% Needed to meet printer requirements.
%
\usepackage{graphics} 	% for pdf, bitmapped graphics files
\usepackage{epsfig} 	% for postscript graphics files
\usepackage{times} 		% assumes new font selection scheme installed
\usepackage{amsmath} 	% assumes amsmath package installed
\usepackage{amssymb}  	% assumes amsmath package installed
\usepackage{graphicx}
\usepackage{caption}
\usepackage{subcaption}
\captionsetup{compatibility=false}
\usepackage{acronym}
\usepackage{bm}
\usepackage{cite}
\usepackage{enumerate}
\usepackage[ruled,vlined,linesnumbered]{algorithm2e}
\let\chapter\section % Necessary for List of TODOs to work. algorithm2e messes up \chapter for some reason...
\usepackage[colorinlistoftodos]{todonotes}
%\usepackage[disable]{todonotes}
\usepackage{hyperref}
%\usepackage{color}
\usepackage{gensymb}
%\usepackage{mathtools}
%
%\setlength{\textfloatsep}{\baselineskip plus 0.2\baselineskip minus 0.2\baselineskip}
%\setlength{\textfloatsep}{5pt}		% Reduces white space between floating figures and the text below them.
%
\DeclareMathOperator{\D}{\rotatebox[origin=c]{45}{$\Box$}}
\DeclareMathOperator{\X}{\bigcirc}
\DeclareMathOperator{\G}{\Box}
\newcommand{\LTLG}{\G}
\newcommand{\LTLF}{\F}
\newcommand{\LTLX}{\X}
%
\acrodef{wrt}[w.r.t.]{with respect to}
%
% Define an "Example" environment with its own counter    
\newcounter{examplecounter}		% Example counter. Independent of theorem and other counters.
\newenvironment{myExample}
{
	\refstepcounter{examplecounter}%
	\textbf{Example \arabic{examplecounter}:}% Example #.
	\,							% Let's give our text some space.
}

% Define a "Definition" environment with its own counter    
\newcounter{definitioncounter}		% Definition counter. Independent of theorem and other counters.
\newenvironment{myDefinition}
{
	\refstepcounter{definitioncounter}%
	\textbf{Definition \arabic{definitioncounter}}%
}

% Define a "Problem" environment with its own counter    
\newcounter{problemcounter}
\newenvironment{myProblem}
{
	\refstepcounter{problemcounter}%
	\textbf{Problem \arabic{problemcounter}}% Definition #.
}

% Define an "Assumption" environment with its own counter    
\newcounter{assumptioncounter}
\newenvironment{myAssumption}
{
	\refstepcounter{assumptioncounter}%
	\textbf{Assumption \arabic{assumptioncounter}:}%
}

%
\title{\LARGE \bf
	Behavior Synthesis for an ATLAS Humanoid Robot\\ from High-level User Specifications
}
\author{Spyros Maniatopoulos, Philipp Schillinger, Vitchyr H. Pong, David C. Conner, and Hadas Kress-Gazit% 	<-this % stops a space
%\thanks{...}% <-this % stops a space
\thanks{This work was supported by \ldots.}
\thanks{
Philipp Schillinger is with BOSCH \ldots {\tt philipp.schillinger\nolinkurl{@de.bosch.com}}.
David C. Conner is with \ldots {\tt davidconner\nolinkurl{@...}}.
The remaining authors are with the Verifiable Robotics Research Group, Autonomous Systems Lab, Sibley School of Mechanical and Aerospace Engineering, Cornell University, Ithaca, NY 14853, USA, {\tt \{sm2296, vhp22, hadaskg\}\nolinkurl{@cornell.edu}}.
}% end thanks
}% end authors
%
%%%%%%%%%%%%%%%%%%%%%%%%%%%%%%%%%%%%%%%%%%%%%%%%%%%%%
%
\begin{document}
%
\maketitle
\thispagestyle{empty}
\pagestyle{empty}
%
%%%%%%%%%%%%%%%%%%%%%%%%%%%%%%%%%%%%%%%%%%%%%%%%%%%%%
\listoftodos[List of TODOs, Fixes, Open Issues]
%
%\setcounter{tocdepth}{2}
%\tableofcontents
\newpage
%%%%%%%%%%%%%%%%%%%%%%%%%%%%%%%%%%%%%%%%%%%%%%%%%%%%%
\begin{abstract}
In this paper we \ldots
\todo[inline, caption = {Title of a TODO item (hyperlink)}]{Body of the TODO item}
\end{abstract}
%
% --- Main sections ----
%
\section{Introduction}\label{S:intro}
% !TEX root = ../main.tex

In preparation for the 2015 DARPA Robotics Challenge (DRC) Finals, Team ViGIR, as well as many other teams, developed an approach to high-level robot control \cite{Philipp2013Bsc, Philipp2015Msc}.
However, these approaches relied on experts developing scripted behaviors or, in the case of Team ViGIR, manually designing state machines.
In addition, there was no guarantee that the resulting high-level behavior was correct \ac{wrt} the task at hand.
Moreover, many participants observed that such approaches were fragile in practice \cite{DRC-what-happened}.

\begin{figure}[t]
\centering
\includegraphics[width=0.99\columnwidth,clip]{./img/atlas_door_finals.png}
\caption{Team ViGIR's Atlas humanoid robot on the first day of the DRC Finals. (Photo credit: DARPA)
}
\label{Fig:AtlasDoorFinals}
\end{figure}

Motivated by these shortcomings, we present an approach for the automatic generation of software that implements high-level robot behaviors in a provably-correct manner.
This was enabled in part by recent advances in the field of formal methods for robotics.
Specifically, correct-by-construction mission plans can be automatically synthesized from high-level, logic-based specifications [cite ALL the papers].
\todo[inline, caption = {Cite a bunch of research groups}]{Don't forget to ``cite ALL the papers here!"}

\begin{myExample}\label{Ex:PickupObject}
	Consider the task, ``Walk to the valve and turn it" (Fig. \ref{Fig:AtlasDoorFinals}).
	\todo[inline, caption = {Example of a ``high-level behavior"}]{Maybe provide example of ``high-level behavior" (identify valve, walk to it, use arm and hand to turn it)}
	This would be an intuitive way to express the task from a non-expert user's point-of-view.
	However, this task specification is not formal, it does not account for the robot's capabilities -- a robot with no means of locomotion or manipulation wouldn't even be able to carry it out -- and it does not specify what should happen if a failure occurs. % manipulation?
\end{myExample}

However, writing a formal, logic-based specification is a non-trivial task that requires expert knowledge.
In this paper, we automatically generate the logic-based specifications from a higher level, partial, multi-paradigm specifications: a description of the system's capabilities, the task's goals, and the task's initial conditions.
Specifically, the system description, which includes constraints such as action preconditions, is set up \emph{a priori} by the system designers.
Then, non-expert users only have to specify a task's goal and initial conditions, without worrying about the internals of the robot and the software it is running.
Furthermore, most approaches in this field assume that the simple, low-level system components that make up the high-level plan will work as expected, i.e., they never fail.
In this paper, we take a first step towards lifting this assumption by formally accounting for the possibility of failure when executing the low-level components.
We achieve this by generalizing the concepts of ``activation" and ``completion", which were introduced by Raman, et al. \cite{Vasu2013ICRA} to deal with the time semantics of logic-based specifications.
While there might be no way to recover from a failure, we can still achieve \emph{graceful degradation}.
That is, we want to specify the system's reaction to failure in the formal specification.

Furthermore, ours is an end-to-end approach that starts with an informal specification and results in an executable software implementation of a high-level plan.
We first create a discrete abstraction of the problem and automatically construct a formal task specification in a fragment of Linear Temporal Logic (\textsc{LTL}).
We then synthesize a \emph{reactive} mission plan that is guaranteed to satisfy the formal specification.
Finally, in an effort to bridge the gap between theoretical results and practice, we automatically generate the implementation of a state machine that instantiates the synthesized plan in software.

In this paper, we present and experimentally validate the proposed approach in the context of a Boston Dynamics Atlas humanoid robot running the software that Team ViGIR developed for the DRC (Fig. \ref{Fig:AtlasDoorFinals}).
However, the concepts apply to different systems.
We have implemented and open-sourced the proposed approach as a collection of Robot Operating System (ROS) packages \cite{ROS2009ICRA, ROS}.

\subsubsection*{Related Work}
Our work generalizes that by Raman, et al. \cite{Vasu2013ICRA} in a way that allows us to reason about multiple outcomes of an action, such as failure, in the formal specification.
\todo[inline, caption = {Dibs on graceful degradation in formal synthesis ?}]{``To the best of our knowledge, this is the first time graceful degradation (or failure) is addressed in the context of formal synthesis" [?]}
Sucan and Kavraki \cite{Kavraki2012ICRA} and DeCastro, et al. \cite{Jon2015ICRA} deal with the related problem of uncertainty in mobile robot motion.
In terms of our approach to creating the formal specification, some other options would have been asking the user to (i) write the formal specification, e.g., \textsc{ltl} formulas, directly, (ii) write Structured English statements \cite{JFRKG2012ICRA}, or (iii) specify the task in natural language \cite{Lignos2015AURO}.
Each option comes with trade-offs and we chose one where the user input is essentially minimal.
In terms of the mission planning step, we opted for \textsc{gr(1)}, i.e., reactive \textsc{ltl}, synthesis \cite{Bloem2012GR1} over other approaches.
These included classical AI planners, such as STRIPS \cite{STRIPS1971AI} and PDDL \cite{PDDL1998TR}, optimization under \textsc{ltl} constraints \cite{Wolff2014ICRA}, and, most notably, synthesis from co-safe \textsc{ltl} specifications (see, e.g., \cite{Kavraki2015ICRA}).
Our main reason for choosing \textsc{gr(1)} synthesis is the ability to specify reactivity \ac{wrt} a dynamic, and even adversarial (worst-case), environment (such as external events and component failures).
Finally, Mehta, et al.\cite{Ankur2015ISRR} also present an end-to-end approach (from formal specification to software generation), while the toolkit developed by Finucane, et al. \cite{Finucane2010IROS} employs an executive that executes the abstract synthesized automaton.
However, in both works, the user has to write a Structured English \cite{JFRKG2012ICRA} specification that exactly maps to \textsc{ltl}, a non-trivial task.
In our work, the user input is a partial, informal specification.
Finally, it should be noted that Filippidis, et al. \cite{Filippidis2015SYNT} also employ \textsc{gr(1)} synthesis from multi-paradigm specifications.
The imperative element of their specification language could have been used in our work, but is not necessary.

The rest of this paper is organized as follows.
In Section \ref{S:prelim}, we introduce Atlas, Team ViGIR's approach to the DRC, and Linear Temporal Logic.
In Section \ref{S:problem}, we state the problems that this paper addresses.
Sections \ref{S:abstraction} through \ref{S:synthesis} present the proposed approach, while Section \ref{S:implementation} summarizes its ROS implementation.
We present experimental demonstrations in Section \ref{S:experiments}.
Finally, we draw conclusions and propose future research directions in Section \ref{S:conclusion}.

%END
%
\section{Preliminaries}\label{S:prelim}
% !TEX root = ../main.tex

\subsection{ATLAS Humanoid Robot}

\ldots

\begin{figure}[t]
\centering
\includegraphics[width=0.99\columnwidth,clip]{./img/control_modes_ts.png}
\caption{
	\todo[inline, caption = {Create a simple control mode TS figure}]{Placeholder! Create simple control mode TS figure (flat).}
}
\label{Fig:ControlModeTS}
%\vspace{-3 pt}
\end{figure}

\subsection{Team ViGIR's Approach to High-level Control}

\ldots

\begin{figure}[t]
\centering
\includegraphics[width=0.99\columnwidth,clip]{./img/behavior_open_door.png}
\caption{A high-level behavior for carrying out the DRC Finals' ``Door" task.
It was designed by Team ViGIR developers using the FlexBE Editor (graphical user interface).
}
\label{Fig:FlexBESM}
\end{figure}

\subsection{Linear Temporal Logic and Reactive LTL Synthesis}

\ldots
%
\section{LTL Specification Compilation}\label{S:ltl}
% !TEX root = ../main.tex

\subsection{Multi-Paradigm Specification}\label{S:multi-paradigm}

Specifying a robot task in a formal language can be a time consuming and error prone process.
It also requires an expert user.
To alleviate these issues, we employ a multi-paradigm specification approach. 
We first observe that there are portions of the task specification $\mathcal{T}_\mathcal{S}$ that are going to be system-specific and portions that are going to be task-specific, such as the task's goals.
Intuitively, a non-expert user should only have to specify the goals $\mathcal{G}$ without worrying about the internals of the robot and the software it is running.
We can infer which actions $\mathcal{A}$ are pertinent to a task and use the discrete abstraction $\mathcal{D}$ as the basis for automatically generating the portion of the formal specification that is related to the system itself.
Finally, the initial conditions $\mathcal{I}$ are either specified by the user or detected at runtime.

Thus, referring to Problem \ref{SpecificationProblem}, we get the goals $\mathcal{G}$ and initial conditions $\mathcal{I}$ from a user.
The discrete abstraction $\mathcal{D}$ is system-specific and has been defined \emph{a priori} by the expert system designers, according to Section \ref{S:abstraction}.
We can now automatically generate the task specification $\mathcal{T}_\mathcal{S}$ in (the \textsc{gr(1)} fragment \cite{Bloem2012GR1} of) Linear Temporal Logic.
Since \textsc{ltl} is compositional, we can generate individual formulas and then conjunct them to get the full \textsc{ltl} specification.


\subsection{Specification of Actions and Control Mode Constraints}

Since the activation of capabilities is controlled by the system, the corresponding \textsc{ltl} formulas will be in $\varphi_s$, the safety requirements (see Section \ref{S:GR1}).
Conversely, we do not control the outcome of activation; the adversarial environment does.
Therefore, the \textsc{ltl} formulas specifying the behavior of outcomes will be in $\varphi_e$, the safety assumptions.

\subsubsection{General Formulas}

We say that an activation proposition $\pi_y$, $y \in \{a, m\}$, is $\True$ when the corresponding primitive capability is being activated and $\False$ when it is not being activated\footnote{Note that this is in contrast to the work of Raman, et al. \cite{Vasu2013ICRA}, where, e.g., $\pi_{camera}$ being $\False$ stands for the act of \emph{deactivating} the corresponding primitive capability, i.e., turning a camera off.}.
Therefore, the system safety requirement \eqref{PropositionDeactivationFormula} dictates that all activation propositions $\pi_y \in \mathcal{Y}$ should turn $\False$ once an outcome has been returned.
Note that the left-hand side of formula \eqref{PropositionDeactivationFormula} is only $\True$ at those distinct time steps where an outcome was just returned.

\todo[inline, caption = {Fix proposition ``deactivation" formula}]{This formula is not consistent with the implementation in ReSpeC, even though they might be 100\% equivalent.
Look like I might be able to remove it altogether!}

\begin{equation}\label{PropositionDeactivationFormula}
	\bigwedge \limits_{o \in Out(y)} \LTLG \Big( \pi_y \wedge \LTLX \pi_y^o \Rightarrow \LTLX \lnot \pi_y \Big)
\end{equation}

The environment safety assumption \eqref{OutcomeMutexFormula} dictates that the outcomes, $\pi_y^o$, of the activation of any system capability are mutually exclusive (e.g., an action cannot both succeed and fail).
Formula \eqref{OutcomeMutexFormula} also allows for no outcome being $\True$.

\begin{equation}\label{OutcomeMutexFormula}
	\bigwedge \limits_{o \in Out(y)} \LTLG \Big( \LTLX \pi_y^o \Rightarrow \bigwedge \limits_{o^\prime \neq o} \LTLX \lnot \pi_y^{o^\prime} \Big)
\end{equation}

The environment safety assumption \eqref{ActionOutcomeConstraintFormula} constraints the value of outcomes.
Specifically, it dictates that, if an outcome is $\False$ and the corresponding capability is not being activated, then that outcome should remain $\False$.
It is a generalization of formula (4) in \cite{Vasu2013ICRA}.

\begin{equation}\label{ActionOutcomeConstraintFormula}
	\bigwedge \limits_{o \in Out(y)} \LTLG \Big( \lnot \pi_y^o \wedge \lnot \pi_y \Rightarrow \LTLX \lnot \pi_y^o \Big)
\end{equation}

%%%%%%%%%%%%%%%%%%%%%%%%%%%%%%%%%%%%%%%%%%%%%%%%%%%%%

\subsubsection{Action-specific Formulas}

The following formulas encode the connection between the activation and the possible outcomes of the system's actions, $a \in \mathcal{A}$.

The environment safety assumption \eqref{ActionOutcomePersistenceFormula} dictates that the value of an outcome should not change if the corresponding action has not been activated again. 
In other words, outcomes persist through time.

\begin{equation}\label{ActionOutcomePersistenceFormula}
	\bigwedge \limits_{o \in Out(a)} \LTLG \Big( \pi_a^o \wedge \lnot \pi_a \Rightarrow \LTLX \pi_a^o \Big)
\end{equation}

The environment liveness assumption \eqref{ActionFairnessConditionsFormula} is a fairness condition.
It states that (always) eventually, the activation of an action will result in an outcome, unless that action is not activated to begin with.
%The disjunct $\lnot \pi_a$ is added in order to prevent situations where the environment loses the game due to the system never activating the action.

\begin{equation}\label{ActionFairnessConditionsFormula}
	\LTLG \LTLF \Big( \Big( \pi_a \wedge \bigvee \limits_{o \in Out(a)} \LTLX \pi_a^o \Big) \vee \lnot \pi_a \Big)
\end{equation}

The system safety requirement \eqref{PreconditionsFormula} constrains the activation of an action $a$ unless its preconditions, $Prec(a)$, are met.

\begin{equation}\label{PreconditionsFormula}
	\LTLG \Big( \bigvee \limits_{y \in Prec(a)} \lnot \pi_y^c \Rightarrow \lnot \pi_a \Big)
\end{equation}
where the superscript $c \in Out(y)$ stands for ``completion".

%%%%%%%%%%%%%%%%%%%%%%%%%%%%%%%%%%%%%%%%%%%%%%%%%%%%%

\subsubsection{Control Mode Formulas}

For brevity of notation, let $$\varphi_m = \pi_m \wedge \bigwedge_{m^\prime \neq m} \lnot \pi_{m^\prime}$$
Activating $\varphi_m$, as opposed to $\pi_m$, takes into account the mutual exclusion between control modes $m \in \mathcal{M}$.
Also let $$\varphi_\mathcal{M}^{none} = \bigwedge_{m \in \mathcal{M}} \lnot \pi_m,$$
where $\varphi_\mathcal{M}^{none}$ being $\True$ stands for not activating any control mode transitions, i.e., staying in the same control mode.

The system safety requirement \eqref{TransitionRelationFormula} encodes the BDI control mode transition system (Section \ref{S:CMActions}, Fig. \ref{Fig:ControlModeTS}) in \textsc{ltl}.

\begin{equation}\label{TransitionRelationFormula}
	\bigwedge \limits_{m \in \mathcal{M}} \LTLG \Big( \LTLX \pi_m^c \Rightarrow \bigvee \limits_{m^\prime \in Adj(m)} \LTLX \varphi_{m^\prime} \vee \LTLX \varphi_\mathcal{M}^{none} \Big)
\end{equation}

The environment safety assumption \eqref{TopologyMutexFormula} enforces mutual exclusion between the BDI control modes.

\begin{equation}\label{TopologyMutexFormula}
	\bigwedge \limits_{m \in \mathcal{M}} \LTLG \Big( \LTLX \pi_m^c \Leftrightarrow \bigwedge \limits_{m^\prime \neq m} \LTLX \lnot \pi_{m^\prime}^c \Big)
\end{equation}

The environment safety assumption \eqref{SingleStepChangeFormula} governs how the active control mode can change (or not) in a single time step, in response to the activation of a control mode transition.

\begin{equation}\label{SingleStepChangeFormula}
	\bigwedge \limits_{m \in \mathcal{M}} \bigwedge \limits_{m^\prime \in Adj(m)} \LTLG \Big( \pi_m^c \wedge  \varphi_{m^\prime} \Rightarrow \big( \LTLX \pi_{m}^c \bigvee \limits_{o \in Out(m^\prime)} \LTLX \pi_{m^\prime}^o \big) \Big)
\end{equation}

Similar to \eqref{ActionOutcomePersistenceFormula}, the environment safety assumption \eqref{TopologyOutcomePersistenceFormula} dictates that the value of the outcomes of control mode transitions must not change if no transition is being activated.

\begin{equation}\label{TopologyOutcomePersistenceFormula}
	\bigwedge \limits_{m \in \mathcal{M}} \bigwedge \limits_{o \in Out(m)} \LTLG \Big( \pi_m^o \wedge \varphi_\mathcal{M}^{none} \Rightarrow \LTLX \pi_m^o \Big)
\end{equation}

The environment liveness assumption \eqref{TopologyFairnessConditionsFormula} is the equivalent of the fairness condition \eqref{ActionFairnessConditionsFormula} for control modes.
A single formula suffices for mutually exclusive propositions \cite{Vasu2013ICRA}.

\begin{equation}\label{TopologyFairnessConditionsFormula}
	\LTLG \LTLF \Big( \bigvee \limits_{m \in \mathcal{M}} \Big( \varphi_m \wedge \bigvee \limits_{o \in Out(m)} \LTLX \pi_m^o \Big) \vee \varphi_\mathcal{M}^{none} \Big)
\end{equation}

This concludes the system-specific portion of $\mathcal{T_S}$.

%%%%%%%%%%%%%%%%%%%%%%%%%%%%%%%%%%%%%%%%%%%%%%%%%%%%%

\subsection{Specification of Task Goals}\label{S:ltl-goals}

Motivated by the DRC tasks, we present formulas that encode the accomplishment of each goal once.
However, \textsc{ltl} can naturally handle repeating tasks (e.g. patrolling).
We can even combine the two paradigms, e.g.,
``Accomplish the goals $\mathcal{G}$ infinitely often, but if anything fails, abort".

The system safety requirements \eqref{MemoryFormula} - \eqref{SMOutcomePersistenceFormula} and liveness requirement \eqref{SuccessLivenessFormula} specify the achievement of the user-provided goals, $g \in \mathcal{G}$, over a finite run (using the same \textsc{ltl} semantics as for infinite execution).
In this paradigm, we say that the execution itself has outcomes too.
We denote them by $o \in Out(Exec)$ and, for simplicity, $Out(Exec) = \{ c, f \}$.
Note that the propositions corresponding to these outcomes, $\pi_{Exec}^o$, are system, not environment, propositions.
We also introduce auxiliary system propositions, $\mu_g$, which serve as memory of having accomplished each goal $g \in \mathcal{G}$.

\begin{equation}\label{MemoryFormula}
	\bigwedge \limits_{g \in \mathcal{G}} \LTLG \big( \LTLX \pi_g^c \vee \mu_g \Leftrightarrow \LTLX \mu_g \big)
\end{equation}

\begin{subequations}
	\label{SMOutcomeFormulas}
	\begin{align}
		\LTLG \Big( \pi_{Exec}^{c} &\Leftrightarrow \bigwedge \limits_{g \in \mathcal{G}} \mu_g \Big) \label{SuccessfulOutcomeFormula}\\
		\LTLG \Big( \pi_{Exec}^{f} &\Leftrightarrow \bigvee \limits_{\pi \in \mathcal{Y}} \pi^f \Big) \label{FailedOutcomeFormula}	 
	\end{align}
\end{subequations}

\begin{equation}\label{SMOutcomePersistenceFormula}
	\bigwedge \limits_{o \in Out(Exec)} \LTLG \Big( \pi_{Exec}^{o} \Rightarrow \LTLX \pi_{Exec}^{o} \Big)
\end{equation}

\begin{equation}\label{SuccessLivenessFormula}
	\LTLG \LTLF \big( \bigvee \limits_{o \in Out(Exec)} \pi_{Exec}^{o} \big)
\end{equation}
The formulas above can be interpreted as: ``If nothing fails, then eventually accomplish each goal. Otherwise, abort".
That is, we assume that the desired reaction to failure $\mathcal{F}$ in Problem \ref{SpecificationProblem} is to stop execution.
While this may sound simplistic and overly conservative, it is actually in line with real-world settings.
For example, NASA JPL's Mars rovers automatically terminate an autonomous drive if the activation of any actuators results in excessive motor current, rover tilt, wheel slip, etc \cite{MER2006Aero}.
Of course this is but one option; the system designers can specify different reactions to failure.

Formula \eqref{MemoryFormula} does not guarantee that the goals will be achieved in a specific order.
However, that is often desirable.
To this end, we can define the goals as an ordered set $\mathcal{G} = \{ g_1, g_2, \ldots, g_n \}$, where $g_i < g_j$ for $i<j$, and the relation $g_i < g_j$ means that goal $g_i$ has to be achieved before $g_j$.
With this definition, we can replace the safety requirement \eqref{MemoryFormula} with \eqref{GoalOrderFormula}, whenever strict goal order is desired.

\begin{equation}\label{GoalOrderFormula}
%	\bigwedge \limits_{i = 1}^n \LTLG \Big(  \lnot \mu_{g_{i-1}} \Rightarrow \LTLX \lnot \mu_{g_i} \Big), \; \mu_{g_0} \triangleq \True
	\bigwedge \limits_{i = 1}^n \LTLG \big( (\pi_{g_i} \wedge \LTLX \pi_{g_i}^c) \wedge \mu_{g_{i-1}} \vee \mu_{g_i} \Leftrightarrow \LTLX \mu_{g_i} \big),
\end{equation}
where $\mu_{g_0} \triangleq \True$.
Formula \eqref{GoalOrderFormula} forces the system to carry out goal $g_i$ after it has accomplished goal $g_{i-1}$.
It can still activate the capability corresponding to $\pi_{g_i}$ earlier, as necessitated by other parts of the task, but that will not count towards achievement of $g_i$ (indicated by $\mu_{g_i}$ being $\True$).

Finally, these auxiliary propositions (memory and outcomes of the run) are added to the system propositions: $$\mathcal{Y} = \mathcal{Y} \cup \bigcup \limits_{g \in \mathcal{G}} \mu_g \cup \bigcup \limits_{o \in Out(Exec)} \pi_{Exec}^o$$

%%%%%%%%%%%%%%%%%%%%%%%%%%%%%%%%%%%%%%%%%%%%%%%%%%%%%

\subsection{Specification of Initial Conditions}

%So far, we have handled the system-specific portion of $\mathcal{T_S}$ and the user-specified task goals, $\mathcal{G}$.
%All that is left is the automatic generation of formulas for the task's initial conditions, $\mathcal{I}$.
For each action $a$ and control mode $m$ in the initial conditions $\mathcal{I}$, the completion proposition should be $\True$ in the environment initial conditions \eqref{EnvironmentInitialConditions}.
All other outcome propositions should be $\False$.

\begin{equation}\label{EnvironmentInitialConditions}
	\varphi_i^e = \bigwedge \limits_{i \in \mathcal{I}} \Big( \pi_i^c \bigwedge \limits_{o \in Out(i)\backslash \{c\}} \lnot \pi_i^o \Big) \wedge \bigwedge \limits_{j \not\in \mathcal{I}} \bigwedge \limits_{o \in Out(j)} \lnot \pi_j^o
\end{equation}

Activation propositions are $\False$ regardless of whether that capability is in $\mathcal{I}$ or not because, if something is already an initial condition, then the resulting plan should not activate it at the beginning of execution.
The auxiliary propositions are also $\False$.
Essentially, all $\pi \in \mathcal{Y}$ are initially $\False$:

\begin{equation}\label{SystemInitialConditions}
	\varphi_i^s = \bigwedge \limits_{i \in \mathcal{I}} \lnot \pi_i \wedge \bigwedge \limits_{j \not \in \mathcal{I}} \lnot \pi_j \bigwedge \limits_{g \in \mathcal{G}} \lnot \mu_g \bigwedge \limits_{o \in Out(Exec)} \lnot \pi_{Exec}^{o}
\end{equation}

%END
%
\section{ROS Implementation}\label{S:implementation}
% !TEX root = ../main.tex

We have implemented all aspects of our approach in $\mathtt{vigir\_behavior\_synthesis}$,\footnote{\scriptsize{\url{https://github.com/team-vigir/vigir_behavior_synthesis}}}
 a collection of Robot Operating System (ROS) Python packages.
Figure \ref{Fig:vigir_behavior_synthesis} depicts these packages as well as the nominal workflow.

\begin{figure}[t]
\centering
\includegraphics[width=0.99\columnwidth,clip]{./img/behavior_synthesis_packages.png}
\caption{
	Team ViGIR's ``Behavior Synthesis" ROS packages and the nominal workflow (clockwise, starting from the left).
}
\label{Fig:vigir_behavior_synthesis}
\end{figure}

The synthesis action server ($\mathtt{vigir\_synthesis\_manager}$) receives a request from the user via FlexBE's GUI.
Given the user's input (initial conditions and goals), the server first requests a full set of LTL formulas for Atlas from the $\mathtt{Generate LTL Specification}$ service ($\mathtt{vigir\_ltl\_specification}$ package).
The generation of the LTL formulas from Section \ref{S:ltl} is delegated to our ``Reactive Specification Construction kit" (ReSpeC),\footnote{\scriptsize{\url{https://github.com/team-vigir/ReSpeC}}}
 which is a Python framework with rudimentary ROS integration.

\begin{figure}[t]
\centering
\fbox{\includegraphics[width=0.90\columnwidth,clip]{./img/synthesis_menu_simple.png}}
\caption{Screenshot of the FlexBE Editor's synthesis menu.
}
\label{Fig:SynthesisMenuSimple}
\end{figure}

The $\mathtt{vigir\_ltl\_synthesizer}$ package acts as a wrapper for external synthesis tools (currently, \cite{SLUGS} is supported).
Given the generated LTL specification, the $\mathtt{Synthesize Automaton}$ service returns a finite-state automaton that is guaranteed to satisfy it, if one exists.
Finally, the server requests a $\mathtt{State Instantiation}$ message from the $\mathtt{Generate FlexBE SM}$ service ($\mathtt{vigir\_sm\_generation}$ package).
This message provides the FlexBE Editor with sufficient information to generate Python code, i.e., an executable state machine that instantiates the synthesized automaton.
The corresponding action, services, and messages are defined in the $\mathtt{vigir\_synthesis\_msgs}$ package.

%\begin{algorithm}[t]
The following excerpt\footnote{We have omitted some details for the sake of brevity and clarity of presentation. For example, most list elements are strings, e.g., \scriptsize{\texttt{"template\_id"}}.}
 is taken from the $\mathtt{State Instantiation}$ list message, which is the end product of the $\mathtt{vigir\_behavior\_synthesis}$ workflow (Fig. \ref{Fig:vigir_behavior_synthesis}).
Specifically, this excerpt corresponds to the primitive functionality $\mathtt{object\_template}$, which appears in Fig. \ref{Fig:stand_and_pick_sm}.

\begin{description}
\setlength{\itemindent}{-.4in}
	\item \scriptsize{\texttt{state\_path: /4\_object\_template}}
	\item \scriptsize{\texttt{state\_class: InputState}}
	\item \scriptsize{\texttt{parameter\_names: [request]}}
	\item \scriptsize{\texttt{parameter\_values: [InputState.SELECTED\_OBJECT\_ID]}}
	\item \scriptsize{\texttt{outcomes: [no\_connection, aborted, received, data\_error]}}
	\item \scriptsize{\texttt{transitions: [failed, failed, 6\_manipulate, failed]}}
	\item \scriptsize{\texttt{autonomy: [0, 0, 0, 0]}}
	\item \scriptsize{\texttt{userdata\_keys: [data]}}
	\item \scriptsize{\texttt{userdata\_remapping: [template\_id]}}
\end{description}

%\caption{Excerpt from $\mathtt{State Instantiation}$ message}
%\label{Alg:StateInstantiation}
%\end{algorithm}

% END
%
\section{Experimental Validation}\label{S:experiments}
% !TEX root = ../main.tex

\todo[inline, caption = {Synthesis time as a function to number of actions ?}]{Provide data on how computationally costly/cheap behavior synthesis is. Time vs number of actions?}
%
\section{Conclusions and Future Work}\label{S:conclusion}
% !TEX root = ../main.tex

In this paper, we presented an end-to-end approach to high-level mission planning.
We combine an informal task specification provided by the user with a discrete abstraction of the robot and software system to automatically generate a formal specification in the \textsc{gr(1)} fragment of Linear Temporal Logic.
We then synthesize a verifiably-correct reactive mission plan.
Finally, we automatically generate a software implementation of the mission plan in the form of an executable state machine.
We implemented our approach as a collection of Robot Operating System packages and experimentally validated it on a Boston Dynamics humanoid robot running the software that Team ViGIR developed for the DARPA Robotics Challenge.

It is important to note that there is a trade-off between expressivity and automation.
On the one hand, an expert user can manually write a very expressive and customized formal specification.
On the other hand, the generation of the formal specification can be automated, as we do here, but possibly at the expense of expressivity (e.g., due to the use of template formulas or hard-coded assumptions.)

The discrete abstraction and formal specification paradigm that we presented in this paper constitute the first steps towards achieving graceful degradation.
In other words, we hinted at the question, ``What does it mean to offer formal guarantees when the activation of robot capabilities can result in failure?"
We plan on further exploring this research direction.

In addition, we are interested in automating another step of our approach, the construct of the discrete abstraction.
Currently, an expert user had to construct it (once for each system).
We believe that by formally specifying the capabilities and requirements of individual system components, we will be able to automatically the discrete abstraction, which includes action preconditions, action outcomes, etc.
Finally, we will be demonstrating our approach on a number of other, more accessible, robotic systems.

% Once we have a synthesized SM, it can be treated as a primitive action with its outcomes, etc.
% Comment on limited message size over bad comms? (send partial specification $\rightarrow$ compile and synthesize onboard)

%\ldots Future work: Capability specification, consider mutex for ``grounding conflicts", integrate more robots, \ldots

%END
%
\section*{Acknowledgments}
The authors would like to thank all other members of Team ViGIR and particularly Alberto Romay and Stefan Kohlbrecher.
%
% ---- Bibliography ----
%
\bibliographystyle{ieeetran}
\bibliography{behavior_synthesis}
%
\end{document}
