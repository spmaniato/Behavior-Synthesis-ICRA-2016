% !TEX root = ../main.tex

\subsection{Control Modes \& Actions}\label{S:CMActions}

We model Atlas' control mode interface (c.f. Fig. \ref{Fig:ControlModeTS}) as a transition system $(\mathcal{M}, \boldsymbol\rightarrow)$, where $\mathcal{M}$ is the set of states, each corresponding to one control mode, $m \in \mathcal{M}$, and $\boldsymbol\rightarrow$ is a set of valid control mode transitions (subset of $\mathcal{M} \times \mathcal{M}$).
In addition, we define $Adj(m) = \{ m^\prime \in \mathcal{M} \; | \; (m, m^\prime) \in \; \boldsymbol\rightarrow \}$ and also allow self-transitions, i.e., $m \in Adj(m), \; \forall m \in \mathcal{M}$.

Furthermore, the system can perform actions. Actions, $a \in \mathcal{A}$, correspond to system capabilities $\mathcal{C}$ (other than control mode changes), e.g., generation of a footstep plan or closing the fingers.
Actions may also have one or more preconditions.
Action preconditions can be control modes or (the completion of) other actions, i.e., $Prec(a) \in 2^{(\mathcal{A} \cup \mathcal{M})}$ (power set) and $a \not \in Prec(a)$, $\forall a \in \mathcal{A}$.
\todo[inline, caption = {What is the Discrete Abstraction exactly?}]{Are actions, preconditions, and the control mode TS part of the discrete abstraction or should they be defined in the Preliminaries instead? What is $\mathcal{D}$ mathematically?}
\todo[inline, caption = {Our preconditions are kinda limited}]{We are not handling preconditions that refer to the state of the \emph{world}, e.g. ``object in hand". Instead, we precondition on (the completion of) actions, e.g., grasping the object, which is not as powerful in my opinion.}

\subsection{Atomic Propositions}

We adopt a paradigm that generalizes the one in \cite{Vasu2013ICRA}.
We abstract the actions, $a \in \mathcal{A}$, that Atlas can perform (activate) using one system proposition, $\pi_a$, per action and one environment proposition, $\pi_a^o$, per possible outcome of that action, $o \in Out(a)$.
Similarly,\footnote{The distinction between action and control mode propositions is purely for the sake of clarity of notation. There is nothing special about either.}
for each control mode, $m \in \mathcal{M}$, we also have an activation proposition $\pi_m$ and a number of outcome propositions $\pi_m^o$.
Therefore, the set of atomic propositions $AP$ is given by Eq. \eqref{ActOutAP}:

\begin{subequations}
	\label{ActOutAP}
	\begin{align}
		\mathcal{Y} &= \bigcup \limits_{a \in \mathcal{A}} \pi_a \bigcup \limits_{m \in \mathcal{M}} \pi_m,\\
%		\mathcal{X}^\prime &= \mathcal{X} 
		\mathcal{X} &= 
		\bigcup \limits_{a \in \mathcal{A}} \bigcup \limits_{o \in Out(a)} \pi_a^o \bigcup \limits_{m \in \mathcal{M}} \bigcup \limits_{o \in Out(m)} \pi_m^o,
	\end{align}
\end{subequations}
%where $\mathcal{X}$ are environment propositions other than outcome propositions, e.g., ones that abstract sensors, as per \cite{KGFP_TRO09}.
It is up to us to decide which possible outcomes of activating a system capability we want to model explicitly.
For the sake of simplicity of presentation, here we will abstract all positive outcomes as ``completion" ($c$) and all negative outcomes as ``failure" ($f$). 
That is, $Out(a) = Out(m) = \{ c, f \}$, $\forall a,m$.
%For both actions and control mode transitions, the outcomes that are of most interest in the context of this paper are completion ($c$) and failure ($f$) of the action. That is, $Out(a) = Out(m) = \{ c, f \}$.

To conclude this section, we could say (somewhat informally) that the resulting discrete abstraction $\mathcal{D}$ consists of symbols ($m \in \mathcal{M}$, $a \in \mathcal{A}$, $\pi \in AP$) and relations between them (e.g., $\boldsymbol\rightarrow$, $Prec$, $Out$).
We defer a discussion of the mapping $\gamma: \mathcal{D} \rightarrow \mathcal{C}$ until Section \ref{S:synthesis}.

%\subsection{Discrete Mapping}
%
%\todo[inline, caption = {Level of detail in definition of $\mathcal{D}_M$}]{If we are going down the generic route, then this is not the right time to relate $\mathcal{D}_M$ to FlexBE. Remove subsection or talk in terms of primitive system capabilities.}
%The mapping $\mathcal{D}_M: \mathcal{D} \rightarrow \mathcal{S}$ from Problem \ref{DiscreteAbstractionProblem} relates parametrized FlexBE state implementations with atomic propositions.
%Specifically, each activation proposition $\pi_p$ is mapped to the execution of a parametrized state implementation $\mathsf{s}_p$.
%In addition, each outcome proposition $\pi_p^o$ is mapped to an outcome $o \in Out(\mathsf{s})$ of the state implementation.
%In practice, an outcome proposition can correspond to multiple outcomes of the state implementation.
%For example, we may want to treat the outcomes $\mathtt{failed}$ and $\mathtt{aborted}$ of $\mathsf{s}_p$ as failure, thus mapping the proposition $\pi_p^f$ to both of them.

% END