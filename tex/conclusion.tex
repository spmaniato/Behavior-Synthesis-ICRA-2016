% !TEX root = ../main.tex

\addtolength{\textheight}{-13 pt}   % This command serves to balance the column lengths
                                  % on the last page of the document manually. It shortens
                                  % the textheight of the last page by a suitable amount.
                                  % This command does not take effect until the next page
                                  % so it should come on the page before the last. Make
                                  % sure that you do not shorten the textheight too much.

We presented an end-to-end approach to mission planning for complex robotic systems.
We combined a task, specified by a non-expert user, with a discrete abstraction of the system, defined \emph{a priori}, to automatically generate a formal specification.
We then synthesized a provably correct mission plan that achieves the task's goals or reacts to any failures in the low-level system components.
Finally, we automatically generated a software implementation of the reactive mission plan in the form of an executable state machine.
We implemented our approach as a collection of ROS packages and experimentally validated it on an Atlas humanoid robot.
%running the software that Team ViGIR developed for the DRC.

It is important to point out the trade-off between expressivity and automation.
On the one hand, an expert user can manually write a very expressive and customized formal specification.
On the other hand, the generation of the formal specification can be automated, as we do here, but possibly at the expense of expressivity (e.g., due to hard-coded design choices.)
However, there is no research barrier to recovering expressivity; the formal language supports it.
Thus, we plan on extending our user interface in immediate future work.

The discrete abstraction and formal specification paradigm that we presented in this paper constitute the first steps towards achieving graceful degradation.
In other words, we hinted at the question, ``What formal guarantees can we offer when the execution of robot capabilities can result in failure?"
We plan on further exploring this research direction.

We are also interested in automating the construction of the discrete abstraction, which includes action preconditions, outcomes, etc.
Currently, the system designer defines it (once per system).
We believe that, by formally specifying the capabilities and constraints of individual system components, we will be able to automatically generate the discrete abstraction.
Finally, we will be demonstrating our approach on a number of different robotic systems in the near future.

%END