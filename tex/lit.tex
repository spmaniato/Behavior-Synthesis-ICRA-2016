% !TEX root = ../main.tex

We opted for automatically constructing the formal specification from user input and an \emph{a priori} description of the system.
Another option would have been asking the user to write the entire formal specification directly, e.g., in \textsc{ltl}, Structured English \cite{JFRKG2012ICRA}, or natural language \cite{Lignos2015AURO}.
Filippidis, et al. \cite{Filippidis2015SYNT} also employ synthesis from multi-paradigm specifications.
The imperative element of their specification language could have benefited our work, but is not necessary.

In terms of prior efforts to achieve graceful degradation in formal synthesis, Lahijanian, et al. \cite{Kavraki2015AAAI} and Kim and Fainekos \cite{Fainekos2014ICRA} consider the partial satisfaction of a formal specification, if it cannot be  completely achieved.
Conversely, Guo and Dimarogonas \cite{Dimos2014ICRA} guarantee the satisfaction of the hard portion of a specification and gradually improve that of the soft portion.
DeCastro, et al. \cite{Jon2015ICRA} deal with the problem of uncertainty in mobile robot motion.
Johnson and Kress-Gazit \cite{Ben2015IJRR} propose a probabilistic framework for analyzing sensor and actuation errors in formal synthesis.
By contrast, we can account for, and specify a reaction to, worst-case failures (and other outcomes) of a real-world system's components.

In terms of the mission planning step, we opted for \textsc{gr(1)} synthesis \cite{Bloem2012GR1}, i.e., reactive \textsc{ltl} synthesis, over other approaches.
These included classical AI planners, such as STRIPS \cite{STRIPS1971AI} and PDDL \cite{PDDL1998TR}, optimization under \textsc{ltl} constraints \cite{Wolff2014ICRA}, and synthesis from co-safe \textsc{ltl} specifications \cite{Kavraki2015AAAI, Kavraki2015ICRA, Belta2014TAC}.
Our main reason for choosing \textsc{gr(1)} synthesis is the ability to specify reactivity with respect to a dynamic, and even adversarial (worst-case), environment, such as external events sensed by the robot and low-level system failures.

Mehta, et al.\cite{Ankur2015ISRR} also present an end-to-end approach (from formal specification to code generation), while the toolkit developed by Finucane, et al. \cite{Finucane2010IROS} employs an executive that executes the synthesized symbolic automaton without the need for code generation.
In our work, the user does not have to input a complete specification.
Most importantly, our approach takes advantage of existing robotics software, namely, ROS \cite{ROS} and its large number of packages.

Finally, we \emph{experimentally} validate our work on a real, complex system, Atlas, whereas most works in formal methods for robotics are only demonstrated in simulation.

% END