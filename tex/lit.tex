% !TEX root = ../main.tex

In terms of prior efforts to achieve graceful degradation in formal synthesis, Lahijanian, et al. \cite{Kavraki2015AAAI} and Kim and Fainekos \cite{Fainekos2014ICRA} consider the partial satisfaction of a formal specification, if the complete one cannot be achieved.
Sucan and Kavraki \cite{Kavraki2012ICRA} and DeCastro, et al. \cite{Jon2015ICRA} deal with the problem of uncertainty in mobile robot motion.
Johnson and Kress-Gazit \cite{Ben2015IJRR} propose a probabilistic framework for analyzing sensor and actuation errors in formal synthesis.
By contrast, we can account for, and specify a reaction to, worst-case failures (and other outcomes) of a real-world system's components.
%our work generalizes that of Raman, et al. \cite{Vasu2013ICRA, Vasu2015TRO} in a way that allows us to reason about multiple outcomes of an action, including failure, in the formal specification in a worst-case manner.


In terms of our approach to creating the formal specification, we opted for automatically constructing it from user-specified tasks and an \emph{a priori} description of the system.
Another option would have been asking the user to write the entire formal specification directly, e.g., in \textsc{ltl}, Structured English \cite{JFRKG2012ICRA}, or natural language \cite{Lignos2015AURO}.
Filippidis, et al. \cite{Filippidis2015SYNT} also employ synthesis from multi-paradigm specifications.
The imperative element of their specification language could have benefited our work, but is not necessary.

In terms of the mission planning step, we opted for \textsc{gr(1)} synthesis \cite{Bloem2012GR1}, i.e., reactive \textsc{ltl} synthesis, over other approaches.
These included classical AI planners, such as STRIPS \cite{STRIPS1971AI} and PDDL \cite{PDDL1998TR}, optimization under \textsc{ltl} constraints \cite{Wolff2014ICRA}, and synthesis from co-safe \textsc{ltl} specifications (see, e.g., the work of He, et al. \cite{Kavraki2015ICRA} and Aydin Gol, et al. \cite{Belta2014TAC}).
Our main reason for choosing \textsc{gr(1)} synthesis is the ability to specify reactivity \ac{wrt} a dynamic, and even adversarial (worst-case), environment (such as external events sensed by the robot and low-level system failures).

Mehta, et al.\cite{Ankur2015ISRR} also present an end-to-end approach (from formal specification to code generation), while the toolkit developed by Finucane, et al. \cite{Finucane2010IROS} employs an executive that executes the synthesized symbolic automaton without the need for code generation.
However, in both works, the user has to write a Structured English \cite{JFRKG2012ICRA} specification that exactly maps to \textsc{ltl}, a non-trivial task.
In our work, the user input is a partial, informal specification.

\todo[inline, caption = {Comment on real-world system vs simulations}]{Real, complex, robotic system and experimental demo (others \cite{Topcu2011RAM}, \ldots more or less any other robotics paper)}

% END