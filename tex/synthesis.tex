% !TEX root = ../main.tex
\todo[inline, caption = {Title of section that solves Problem 3}]{Maybe rename this to ``Synthesis of Executable State Machines"? (Will depend on whether the term remains overloaded due to FlexBE ``behaviors".)}


We tackle Problem \ref{BehaviorSynthesisProblem} in two sequential steps.
First, we automatically generate a correct-by-construction automaton from the formal specification $\mathcal{T}_S$ using GR(1) synthesis (see \cite{Bloem2012GR1} and Section \ref{S:GR1}).
Specifically, we employ the synthesis algorithm in \cite{SLUGS}, which can handle a slightly larger fragment of LTL than GR(1).
Namely, the one that includes $\LTLX$ (next) operators in liveness formulas, such as in formulas \eqref{ActionFairnessConditionsFormula} and \eqref{TopologyFairnessConditionsFormula}.
This algorithm was first used in \cite{Vasu2013ICRA}.
\todo[inline, caption = {Properly mention SLUGS' fragment}]{@HKG, is it true that Vasu's paper was the first case of synthesis for this fragment? Had Ruediger used it before?}

\todo[inline, caption = {Relate $\gamma$ to FlexBE states}]{Say that, in our setup, the capabilities $\mathcal{C}$ are accessed via parametrized FlexBE states $\mathsf{S_P}$. Therefore \ldots}
Second, we use the mapping $\gamma: \mathcal{D} \rightarrow \mathcal{C}$ to instantiate the abstract automaton as a concrete software implementation, i.e., an executable state machine in the FlexBE framework introduced in Section \ref{S:FlexBE}.
\todo[inline, caption = {Expand more on automaton-to-software (grounding)}]{This needs more explaining, both in the intro and here. This is basically solving the grounding problem of the propositions to code. You need to motivate why this is important, why it is difficult and that this is a critical step to going from the theory of synthesis to deployment on real systems. This should be a main point in this paper. (HKG)}
\todo[inline, caption = {Pre-made states machines can be capabilities too}]{Point out that pre-made state machines (aka behaviors) are system capabilities too. Just not primitive. Notation?}
\todo[inline, caption = {Memory props not in FlexBE SM}]{Memory propositions are not mapped to any system capabilities and therefore do not appear in FlexBE SMs.}

% END